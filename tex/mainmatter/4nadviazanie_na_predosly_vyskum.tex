\chapter{Výsledky a ich náväznosť na predošlý výskum}


Cieľom tejto kapitoly bude, ako jej názov napovedá, prepojiť výsledky z článkov napísaných v minulosti na naše výsledky. Pôjde konkrétne o článok Domaratzkého, Kismana a Shallita \cite{shallit}, ktorí sa zaoberali okrem iného enumeráciou regulárnych jazykov akceptovanými n-stavovými NKA, z čoho, podobne ako my, dostali aj výsledky o deterministickej a nedeterministickej stavovej zložitosti týchto regulárnych jazykov.

\paragraph{}
Zadefinujme nasledovnú funkciu:
\\
$G_k(n)$ = počet rôznych regulárnych jazykov nad k-znakovou abecedou akceptovaných NKA s n stavmi.
\paragraph{}
My sme skúmali predovšetkým $G_2(n)$, ale po menšej úprave programu sa dali rátať i $G_1(n)$ a $G_3(n)$. Vo všetkých prípadoch sa nám podarilo zreprodukovať výsledky z \cite{shallit} a k tomu navyše sa nám podarilo dorátať $G_2(4)$. Získané výsledky uvádzame v nasledujúcich tabuľkách:

\begin{table}[h]
  \centering
  \begin{tabular}{|l|c|c|c|c|r|}
    \hline
    n & 1 & 2 & 3 & 4 & 5 \\ 
    \hline
    $G_1(n)$ & 3 & 9 & 29 & 88 & 269 \\ 
    \hline
  \end{tabular}
  \caption{$G_1(n)$ pre $1 \leq n \leq 5$}
  \label{tab:G1n}
\end{table}

V nasledujúcej tabuľke j označuje počet stavov minimálnych DKA s j stavmi pre nájdené jazyky, ktoré akceptujú NKA s n stavmi:
\\
\\

\begin{table}[h]
  \centering
  \begin{tabular}{|l|c|c|c|c|c|c|c|c|c|c|c|c|c|c|c|c|c|c|r|}
    \hline
    n/j & 1 & 2 & 3 & 4 & 5 & 6 & 7 & 8 & 9 & 10 & 11 & 12 & 13 & 14 & 15 & 16 & 17 & 18 \\ 
    \hline
    1 & 2 & 1 & & & & & & & & & & & & & & & &\\ 
    \hline
    2 & 2 & 4 & 3 & & & & & & & & & & & & & & &\\
    \hline
    3 & 2 & 4 & 12 & 7 & 3 & 1 & & & & & & & & & & & &\\
    \hline
    4 & 2 & 4 & 12 & 30 & 16 & 11 & 8 & 2 & 1 & 1 & 1 & & & & & & &\\
    \hline
    5 & 2 & 4 & 12 & 30 & 78 & 33 & 27 & 29 & 23 & 9 & 6 & 6 & 2 & 3 & 2 & 1 & 1 & 1 \\
    \hline
  \end{tabular}
  \caption{distribúcia minimálnych DKA pre nájdené minimálne NKA pre $G_1(n)$}
  \label{tab:G1n}
\end{table}

\paragraph{}
Teraz si uvedieme výsledky pre $G_2(n)$, ktoré boli aj jedným z hlavných cieľov tejto práce:

\begin{table}[h]
  \centering
  \begin{tabular}{|l|c|c|c|c|r|}
    \hline
    n & 1 & 2 & 3 & 4 \\ 
    \hline
    $G_2(n)$ & 5 & 213 & 45 113 & 32 191 450 \\ 
    \hline
  \end{tabular}
  \caption{$G_2(n)$ pre $1 \leq n \leq 4$}
\end{table}

Priemerná deterministická zložitosť jazykov s nedeterministickou zložitosťou 4 vychádza približne 9.588. A toto je distribúcia minimálnych DKA a NKA pre nájdené jazyky (tabuľku sme otočili o $90\,^{\circ}$ oproti predošlým, aby sa zmestila):

\begin{table}[H]
  \centering
  \begin{tabular}{|l|c|c|c|c|c|r|}
    \hline
    j/n & 1 & 2 & 3 & 4 \\
    \hline
    1 & 2 & 2 & 2 & 2 \\
    \hline
    2 & 3 & 24 & 24 & 24 \\
    \hline
    3 & & 117 & 1 028 & 1 028 \\
    \hline
    4 & & 70 & 5 595 & 56 014 \\
    \hline 
    5 & & & 11 211 & 316 858 \\
    \hline
    6 & & & 14 537 & 1 037 248 \\
    \hline
    7 & & & 10 580 & 2 846 095 \\
    \hline
    8 & & & 2 136 & 5 293 858 \\
    \hline
    9 & & & & 6 744 831 \\
    \hline
    10 & & & & 6 334 902 \\
    \hline
    11 & & & & 4 414 937 \\
    \hline
    12 & & & & 2 707 073 \\
    \hline
    13 & & & & 1 472 277 \\
    \hline
    14 & & & & 606 946 \\
    \hline
    15 & & & & 301 041 \\
    \hline
    16 & & & & 58 316 \\
    \hline
  \end{tabular}
  \caption{distribúcia minimálnych DKA pre nájdené minimálne NKA pre $G_2(n)$}
\end{table}

Prikladáme tiež tabuľku, kde na základe predošlých výsledkov sú k nedeterministickým zložitostiam jazykov vyrátané ich priemerné deterministické zložitosti:

\begin{table}[H]
  \centering
  \begin{tabular}{|l|c|c|c|c|r|}
    \hline
    nedeterministická  zložitosť & 1 & 2 & 3 & 4 \\ 
    \hline
    priemerná det. zl. (približne) & 1.6 & 3.2355 & 5.7741 & 9.5885 \\ 
    \hline
  \end{tabular}
  \caption{Priemerné deterministické zložitosti pre jazyky s danými nedeterministickými zložitosťami}
\end{table}


Ešte uvedieme časy - koľko nám jednotlivé výsledky pre $G_2(n)$ trvali vyrátať na stroji, ktorý sme mali k dispozícii (Intel Core i7 ~3GHz, 4 jadrá, 24GB RAM):

\begin{table}[H]
  \centering
  \begin{tabular}{|l|c|c|c|c|r|}
    \hline
    n & 1 & 2 & 3 & 4 \\ 
    \hline
    $čas$ & <1s & <1s & 29s & 168h 30m (7 dní) \\ 
    \hline
  \end{tabular}
  \caption{časy výpočtu $G_2(n)$ pre $1 \leq n \leq 5$}
\end{table}

Treba poznamenať, že nakoľko sme úlohu neparalelizovali, tak celý čas sa využíval výkon iba jedného jadra CPU.