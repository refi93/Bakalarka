\chapter{Záver}

Podarilo sa nám zistiť deterministickú zložitosť regulárnych jazykov s nedeterministickou zložitosťou 4. Ukázalo sa, že nárast počtu stavov bol v priemere len o niečo viac ako 2-násobný. Pre vzorku jazykov s nedeterministickou zložitosťou 5 vyšiel nárast dokonca menej ako 2-násobný. Na základe výsledkov získaných rovnakou metódou pre jazyky s nedeterministickou zložitosťou 4 však usudzujeme, že v priemere pre všetky jazyky s nedeterministickou zložitosťou 5 to bude zrejme väčšie číslo. Skreslenie výsledkov bolo pravdepodobne spôsobené tým, že sme generovali rovnomerne vzhľadom na priestor automatov, nie jazykov a že s vyššou pravdepodobnosťou sa generovali NKA s nižšou deterministickou zložitosťou.

\paragraph{}
Táto práca zároveň nastoľuje ďalšie otázky. Okrem zistenia deterministickej zložitosti jazykov s vyššími nedeterministickými zložitosťami vyvstáva otázka, aké vlastnosti by mohli mať regulárne jazyky, ktoré majú jednoznačný minimálny NKA. Podarilo sa nám určiť počet týchto jazykov do nedeterministickej zložitosti 3. Ďalej  by sme sa mohli pýtať, na aké operácie sú uzavreté, koľko ich je v pomere so všetkými regulárnymi jazykmi, či existuje pre ne polynomiálny algoritmus, ktorý by našiel minimálny NKA a podobne.