\chapter*{Úvod}
\paragraph{}


Konečné automaty majú uplatnenie vo viacerých oblastiach informatiky, napríklad pri lexikálnej analýze, hľadaní výskytov reťazcov v texte, vytváraní jednoduchých stratégii a inde.

Je známe, že hoci výpočtová sila deterministických konečných automatov (DKA) je ekvivalentná výpočtovej sile nedeterministických konečných automatov (NKA), minimálny DKA môže vzhľadom na ekvivalentný minimálny NKA mať potenciálne až $2^n$ stavov, kde $n$ je počet stavov pôvodného NKA. Pomerne často sa ale môžeme stretnúť s prípadmi, v ktorých je tento nárast značne menší.

Výhodou NKA, ako sme už spomenuli, je menší počet stavov a jednoduchší návrh, keďže tam nie sú také striktné požiadavky. Naproti tomu, DKA sa jednoduchšie implementujú, keďže to lepšie zodpovedá spôsobu, akým fungujú súčasné počítače a v konečnom dôsledku aj väčšina programovacích jazykov. Aj preto by nás mohlo zaujímať, aké je to v skutočnosti ,,zlé'' alebo ,,dobré'' s počtom stavov pri DKA. 

Naším cieľom bude tento nárast počtu stavov v priemernom prípade experimentálne odmerať pre NKA do 4 stavov nad binárnou abecedou a pokúsiť sa nahliadnuť, ako to vyzerá pri väčšom počte stavov a iných abecedách.

Druhá, nemenej zaujímavá, otázka, na ktorú sa pokúsime odpovedať, bude, koľko jedinečných regulárnych jazykov sú schopné akceptovať nedeterministické automaty ohraničené malým počtom stavov nad malou abecedou. Bude to jeden z výsledkov, ktorý vyplynie zo spôsobu, akým budeme náš experiment vykonávať.

V minulosti sa touto otázkou zaoberali napr. v článku \cite{shallit}. Ich výsledky pokryli NKA do 3 stavov. Pokúsime sa ísť trocha ďalej a zistiť situáciu pre viac stavov, čo by nám mohlo poskytnúť lepší odhad, ako to môže byť vo všeobecnosti. Najťažšie (a tiež najzaujímavejšie) bude navrhnúť algoritmus, ktorý je schopný skutočne overiť všetky možnosti na súčasnom hardvéri.
