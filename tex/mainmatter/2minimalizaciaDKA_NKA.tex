\chapter{Minimalizácia DKA, NKA}

\label{brzozowski}
\section{Minimalizácia DKA}

\subsection{Brzozowského algoritmus}
Je to jeden z algoritmov, pomocou ktorého dokážeme k danému DKA A nájsť minimálny DKA B akceptujúci ten istý jazyk. Samotný algoritmus pracuje veľmi jednoducho, ale najprv si definujme, resp. popíšme základne operácie, ktoré budeme na jeho realizáciu využívať:

\begin{defn}{\textbf {\textit {Reverz konečného automatu}}} $A=(K_A,\Sigma,\delta_A,q_{0},F_A)$ je konečný automat $R(A)=(K_A,\Sigma,\delta_{R(A)},F_A,q_{O})$, pre ktorý platí:
\\
\centerline{$\forall q_a,q_b \in K_A: (q_a,a) \rightarrow q_b \in \delta_A \Rightarrow (q_b,a)\rightarrow(q_a)$}
\end{defn}
Je dôležité si všimnúť, že sme zamenili aj počiatočný stav s množinou akceptačných stavov. Tento automat teda nie je NKA v pravom slova zmysle, keďže sa pripúšťa viac počiatočných stavov. Pre formálnu korektnosť nie je ale problém pridať pomocný počiatočný stav, z ktorého by viedol epsilonový prechod na každý z možných pôvodných počiatočných stavov.
\paragraph{}
Samotný algoritmus teraz bude vyzerať nasledovne:\\
Na vstupe máme DKA $A=(K_A,\Sigma,\delta_A,q_{0A},F_A)$. Nech operácia $R(A)$ označuje reverz automatu A a operácia $D(A)$ determinizáciu automatu A, tak ako boli vyššie definované. Potom minimálny DKA B akceptujúci rovnaký jazyk ako NKA A skonštruujeme nasledovne:
\\
\centerline{$B=D(R(D(R(A))))$}
Operácia determinizácie môže mať až exponenciálnu časovú zložitosť vzhľadom na počet stavov pôvodného NKA, keďže toľko stavov môže teoreticky mať DKA ekvivalentný k danému NKA, ako aj naznačovala vyššie uvedená konštrukcia na determinizáciu NKA, avšak v priemernom prípade sa to nejaví byť až také zlé. Takúto časovú zložitosť v konečnom dôsledku dosahuje aj Brzozowského algoritmus, keďže využíva túto operáciu. I keď existujú aj asymptoticky rýchlejšie algoritmy na minimalizáciu NKA, menovite Hopcroftov algoritmus s časovou zložitosťou $O(nslog(n))$, kde $n$ je počet stavov pôvodného DKA a $s$ počet znakov abecedy, ukázalo sa prekvapivo, že aj takýto jednoduchý, i keď pomalší algoritmus bude postačovať pre naše potreby


\paragraph{}
Uvedieme si ešte pre úplnosť ďalší, efektívnejší algoritmus na minimalizáciu DKA, i keď sme ho napokon neimplementovali:

\subsection{Hopcroftov algoritmus}
Algoritmus si uvedieme podobne, ako je to v \cite{petruchova}.
\begin{defn}{\textbf {\textit {Rozklad}}} množiny K tvoria po dvojiciach disjunktné množiny $K_1,K_2,...,K_N$ (tzv. triedy rozkladu) také, že ich zjednotenie tvorí práve množinu K. \end{defn}

\begin{defn}{\textbf {\textit {Kongruencia automatu}}} A je je rozklad množiny stavov K taká, že pre každú triedu rozkladu $K_i$ platí:
\\
\centerline{$\forall p,q \in K, a \in \Sigma: (p,q \in K_i) \Rightarrow (\delta(p,a),\delta(q,a) \in K_j)$}
\end{defn}

\begin{defn}Rozklad $\{K_1,K_2,...,K_n\}$ pokladáme za hrubší ako rozklad $\{K_1',K_2',...,K_m'\}$, pokiaľ každá z tried $K_1,...,K_m$ je zjednotením niekoľkých tried z $K_1',...,K_m'$ \end{defn}

Keď sme si už definovali potrebné pojmy, prejdime na samotný algoritmus. Na začiatku budeme uvažovať dve základné triedy rozkladu množiny stavov automatu A, a to množinu akceptačných a tých ostatných, teda neakceptačných stavov, teda množiny $F$ a $F^C$. Tento rozklad budeme postupne zjemňovať pokiaľ to len bude možné a takto dostaneme najhrubšiu kongruenciu automatu A, ktorá definuje stavy hľadaného minimálneho DKA.

\paragraph{}
Máme rozklad P a množinu dvojíc S = (C,a), kde $a \in \Sigma$ a C je trieda rozkladu P. Budeme pracovať s dvojicami v množine S. Na začiatku $P=\{F,F^C\}$ a $S = (C,a) \forall a \in \Sigma$, kde C je menšia z tried F, F^C. Algoritmus je potom nasledovný:

Vyberieme jednu dvojicu (C,a) z S a pre každú triedu B rozkladu P vykonáme nasledujúce:
\begin{itemize}
  \item Overíme, či $(C,a)$ rozdeľuje B na nové triedy rozkladu, t.j. či platí, že:
  \\
  \centerline{$\exists p,q \in B: \delta(p,a) \in C \wedge \delta(q,a) \not\in C$}
  \\ Ak (C,a) nerozdeľuje B, tak prejdeme na ďalšiu triedu B. Ak (C,a) rozdeľuje B, vytvoríme nové triedy $B_1 = \{q|\delta(q,a) \in C\}$ a $B_2 = \{q|\delta(q,a) \not\in C\}$.
  \item  Triedu B v P zrušíme a nahradíme ju triedami $B_1$ a $B_2$. Teda pre každé $b \in \Sigma$ ak $(B,b) \in S$, tak nahradíme to pármi $(B_1,b)$ a $(B_2,b)$ Ak sa $(B,b)$ v S nenachádza, vyberieme menšiu z množín $B_1$, $B_2$, nech je to $B'$, a pridáme do S dvojicu $(B',b)$.
\end{itemize}
Tieto kroky cyklicky opakujeme, až sa množina S vyprázdni. Rozklad P bude reprezentovať stavy výsledného minimálneho DKA a jeho $\delta$-funkcia bude zasa množina S. Akceptačné stavy budú tie triedy, ktoré sú podmnožinami triedy F. Časová zložitosť algoritmu je $O(ns log n)$ kde n je počet stavov DKA na v stupe a s je veľkosť abecedy, ktorú pre naše potreby môžeme pokladať za konštantnú, čiže reálne máme časovú zložitosť $O(n log n)$. Dôkaz správnosti tohto algoritmu je uvedený v \cite{hudakova}.


\section{Minimalizácia NKA}

Našou úlohou je nájsť NKA s najmenším možným počtom stavov, ktorý akceptuje rovnaký jazyk ako NKA na vstupe.
\paragraph{}
Narozdiel od minimálneho DKA, minimálnych NKA môže existovať viacero rôznych takých, že akceptujú ten istý jazyk.
Navyše, ukazuje sa, že problém minimalizácie NKA je PSPACE-úplný (\cite{jiang}), t.j. ľubovoľný problém riešiteľný s polynomiálne veľkou pamäťou sa dá naň v polynomiálnom čase transformovať, čo nám napovedá, že to je zrejme veľmi ťažký problém. 
Nakoľko problém minimalizácie NKA je PSPACE úplný, dá sa zostrojiť algoritmus riešiaci tento problém využijúc polynomiálne veľa pamäte - $O(n^k)$, ale potrebuje na to exponenciálny čas - $O(2^{n^k})$. Uveďme algoritmus, ktorý to robí v uvedenom čase a s uvedenou pamäťou:
\paragraph{}
Na vstupe máme NKA $A=(K_A,\Sigma,\delta_A,q_{0A},F_A)$ a ideme rozhodnúť, či ide o minimálny NKA pre L(A).
\\
\begin{enumerate}
  \item K NKA A zostrojíme ekvivalentný minimálny DKA D(A).
  \item Vygenerujeme NKA B s menej ako $|K_A|$ stavmi nad abecedou $\Sigma$ - tých je len konečne veľa.
  \item Zostrojíme minimálny DKA D(B) ekvivalentný s B a overíme, či je ekvivalentný s D(A). Keďže D(A) aj D(B) sú minimálne DKA, stačí nájsť izomorfizmus medzi stavmi.
  \item Ak sú D(A) a D(B) ekvivalentné, t.j. existuje medzi množinami stavov izomorfizmus tak prehlásime, že A nie je minimálny NKA, inak sa vrátime ku kroku 2 a vygenerujeme ďalší NKA v poradí.
  \item Pokiaľ sme vygenerovali všetky možné NKA B, a nenašli sme ekvivalentný k NKA A, tak vieme, že NKA A je minimálny.
\end{enumerate}

\paragraph{Časová zložitosť: }
NKA nad fixnou abecedou do n stavov je $O(2^{n^k})$, ako si aj ukážeme v nasledujúcej kapitole (\ref{genNKA}), z čoho aj vyplýva časová zložitosť.

\paragraph{}
Podobný algoritmus až na niektoré optimalizácie budeme využívať aj pri našom experimente. Takto zároveň s generovaním NKA B postupne enumerujeme všetky jazyky akceptované NKA do daného počtu stavov.