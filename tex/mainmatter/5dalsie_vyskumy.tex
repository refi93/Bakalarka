\chapter{Ďalšie výskumy}

V tejto kapitole sa budeme zaoberať ďalšími otázkami, ktoré síce nie sú hlavným predmetom tejto práce, ale odpovede na ne sme dostali ako vedľajší produkt nášho výskumu.

\section{NKA do 5 stavov}
Keď už máme zistené jazyky akceptované NKA do 4 stavov, mohli by sme sa pýtať ako je to pri NKA s 5 stavmi. Aj keby sme zapojili "do hry" optimalizácie, ktoré nám pomohli pri NKA do 4 stavov, zrejme by to trvalo neúnosne dlho. Čo ale môžeme urobiť je, že si vezmeme všetky nájdené jazyky akceptované NKA do 4 stavov a budeme veľakrát generovať náhodné 5-stavové NKA. Následne zistíme, koľkokrát sa nám po vygenerovaní náhodného 5-stavového NKA podarí natrafiť na nový jazyk. Takto môžeme dostať aspoň približný odhad počtu jazykov akceptovaných NKA do 5 stavov.

TODO
\label{safeWordLength}
\section{Dĺžka slov, ktoré jednoznačne odlíšia dva NKA}


\subsection{Popis problému}
Máme dané 2 NKA, ktoré majú oba najviac n stavov. Otázka znie: ,,Slová do akej dĺžky potrebujeme overiť, aby sme mohli s istotou povedať, že tieto NKA sú ekvivalentné alebo nie?''. Túto dĺžku budeme označovať $\delta_n$. Treba ešte poznamenať, že rátame s tým, že overujeme všetky možné slová do tejto dĺžky.  Ak by $\delta_n$ bola ,,rozumne'' malá, tak by sa to napr. dalo použiť ako pomerne účinný test ekvivalencie dvoch NKA iba s tým, žeby sme overili, ktoré slová do tejto dĺžky oba NKA akceptujú a ak sa obe množiny rovnajú, tak by sme vedeli, že sú ekvivalentné, v opačnom prípade nie. A keby to aj tak nebolo, mohlo by nás to zaujímať čisto zo zvedavosti. 

\subsection{Použitá metóda}
Ako sme predtým spomínali pri metódach hashovania, konkrétne pri hashovaní podľa slov do fixnej dĺžky (\ref{hashSlova}), túto metódu môžeme využiť práve na vyrátanie tejto hodnoty. Jediné, čo budeme robiť je, že zahashujeme každý vygenerovaný NKA týmto hashom a následne overíme pre všetky automaty s rovnakým hashom, či niektorý z nich je alebo nie je ekvivalentný tomuto NKA. Postupne budeme hashovacej funkcii navyšovať jej parameter (t.j. dĺžku, po ktorú overuje všetky slová, či ich daný NKA akceptuje) a zastavíme, keď v každom ,,chlieviku'' bude nanajvýš jeden automat. Veľkosť tohto parametra bude naša hľadaná dĺžka, ktorá jednoznačne odlíši dve NKA.

\subsection{Výsledky}

Veľkosť $\delta_n$ sme skúmali pre n-stavové NKA nad dvojznakovou abecedou do 3 stavov, pre väčšie NKA sa už vyššie uvedená metóda javí byť neefektívna.

\begin{table}[h]
  \centering
  \begin{tabular}{|l|c|c|c|r|}
    \hline
    n & 1 & 2 & 3 \\ 
    \hline
    $\delta_n$ & 1 & 4 & 11 \\ 
    \hline
  \end{tabular}
  \caption{$\delta_n$ pre n-stavové  NKA nad 2-znakovou abecedou do 3 stavov}
\end{table}

Ako vidno, nie je to veľmi efektívne, $\delta_n$ pre väčšie n bude rásť zrejme dosť prudko, veď už len pre 3-stavové NKA by sme potreboval overovať $2^12 - 1$, čiže 2047 slov, čo je síce ešte únosné, ale to znamená, že pre väčšie NKA to zrejme budú už milióny až miliardy, čo je zjavne menej efektívne ako riešiť to konvenčne.
\paragraph{}
Takže na jednoznačné overenie to zjavne nestačí. Mohlo by nám to ale poslúžiť možno aspoň pravdepodobnostne - pozrime sa, koľko kolízii vzniklo pri jednotlivých nastaveniach dĺžky slova, ktorú sme overovali - budeme ju označovať $\delta$.

Pozrime sa najprv na NKA do 2 stavov:
\begin{table}[h]
  \centering
  \begin{tabular}{|l|c|c|c|c|c|r|}
    \hline
    $\delta$ & 0 & 1 & 2 & 3 & 4 \\ 
    \hline
    \# kolízii & 124 & 118 & 70 & 7 & 0 \\
    \hline
  \end{tabular}
  \caption{počet kolízii pre jednotlivé dĺžky slov pre NKA do 2 stavov}
\end{table}

A pre NKA do 3 stavov to vyzerá nasledovne:
\begin{table}[h]
  \centering
  \begin{tabular}{|l|c|c|c|c|c|c|c|c|c|c|c|r|}
    \hline
    $\delta$ & 0 & 1 & 2 & 3 & 4 & 5 & 6 & 7 & 8 & 9 & 10 & 11 \\ 
    \hline
    \# kolízii & 28 674 & 28 668 & 28 554 & 24 422 & 12 655 & 3 602 & 1 057 & 232 & 45 & 6 & 2 & 0\\ 
    \hline
  \end{tabular}
  \caption{počet kolízii pre jednotlivé dĺžky slov pre NKA do 3 stavov}
\end{table}

Dalo by s týmto samozrejme ,,hrať'' aj ďalej, napr. zistiť, ako sú rozdistribuované kolízie medzi jednotlivými množinami akceptovaných slov, alebo hľadať, ktoré slová sa vyskytujú najčastejšie/najmenej často medzi tými akceptovanými. Nie je to ale hlavný predmet tejto práce, preto sa tomu nebudeme venovať až do takých detailov. Ak by ale čitateľ bol zvedavý, výstupom tejto práce je aj naimplementovaný program, ku ktorému je v nasledujúcej kapitole aj dokumentácia a môže si, okrem iného, tieto testy sám naprogramovať.


