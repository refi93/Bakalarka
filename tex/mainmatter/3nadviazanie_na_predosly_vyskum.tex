\chapter{Výsledky a ich náväznosť na predošlý výskum}


Cieľom tejto kapitoly bude, ako jej názov napovedá, prepojiť výsledky z článkov napísaných v minulosti na naše výsledky. Pôjde konkrétne o článok Domaratzkého, Kismana a Shallita \cite{shallit}, ktorí sa zaoberali okrem iného enumeráciou regulárnych jazykov akceptovanými n-stavovými NKA, z čoho, podobne ako my, dostali aj výsledky o deterministickej a nedeterministickej stavovej zložitosti týchto regulárnych jazykov.

\paragraph{}
Zadefinujme nasledovnú funkciu:
\\
$G_k(n)$ = počet rôznych regulárnych jazykov nad k-znakovou abecedou akceptovaných NKA s n stavmi.
\paragraph{}
My sme skúmali predovšetkým $G_2(n)$, ale po menšej úprave programu sa dali rátať i $G_1(n)$ a $G_3(n)$. Vo všetkých prípadoch sa nám podarilo zreprodukovať výsledky z \cite{shallit} a k tomu navyše sa nám podarilo dorátať $G_2(4)$. Získané výsledky uvádzame v nasledujúcich tabuľkách:

\begin{table}[h]
  \centering
  \begin{tabular}{|l|c|c|c|c|r|}
    \hline
    n & 1 & 2 & 3 & 4 & 5 \\ 
    \hline
    $G_1(n)$ & 3 & 9 & 29 & 88 & 269 \\ 
    \hline
  \end{tabular}
  \caption{$G_1(n)$ pre $1 \leq n \leq 5$}
  \label{tab:G1n}
\end{table}

V nasledujúcej tabuľke j označuje počet stavov minimálnych DKA s j stavmi pre nájdené jazyky, ktoré akceptujú NKA s n stavmi:
\\
\\

\begin{table}[h]
  \centering
  \begin{tabular}{|l|c|c|c|c|c|c|c|c|c|c|c|c|c|c|c|c|c|c|r|}
    \hline
    n/j & 1 & 2 & 3 & 4 & 5 & 6 & 7 & 8 & 9 & 10 & 11 & 12 & 13 & 14 & 15 & 16 & 17 & 18 \\ 
    \hline
    1 & 2 & 1 & & & & & & & & & & & & & & & &\\ 
    \hline
    2 & 2 & 4 & 3 & & & & & & & & & & & & & & &\\
    \hline
    3 & 2 & 4 & 12 & 7 & 3 & 1 & & & & & & & & & & & &\\
    \hline
    4 & 2 & 4 & 12 & 30 & 16 & 11 & 8 & 2 & 1 & 1 & 1 & & & & & & &\\
    \hline
    5 & 2 & 4 & 12 & 30 & 78 & 33 & 27 & 29 & 23 & 9 & 6 & 6 & 2 & 3 & 2 & 1 & 1 & 1 \\
    \hline
  \end{tabular}
  \caption{distribúcia minimálnych DKA pre nájdené minimálne NKA pre $G_1(n)$}
  \label{tab:G1n}
\end{table}

\paragraph{}
Teraz si uvedieme výsledky pre $G_2(n)$, ktoré boli aj jedným z hlavných cieľov tejto práce:

\begin{table}[h]
  \centering
  \begin{tabular}{|l|c|c|c|c|r|}
    \hline
    n & 1 & 2 & 3 & 4 \\ 
    \hline
    $G_1(n)$ & 5 & 213 & 45 113 & 32 191 450 \\ 
    \hline
  \end{tabular}
  \caption{$G_1(n)$ pre $1 \leq n \leq 5$}
  \label{tab:G1n}
\end{table}

A toto je distribúcia minimálnych DKA a NKA pre nájdené jazyky:

\begin{table}[h]
  \centering
  \begin{tabular}{|l|c|c|c|c|c|c|c|c|c|c|c|c|c|c|c|c|c|c|r|}
    \hline
    n/j & 1 & 2 & 3 & 4 & 5 & 6 & 7 & 8 \\ 
    \hline
    1 & 2 & 3 & & & & & &\\ 
    \hline
    2 & 2 & 24 & 117 & 70 & & & &\\
    \hline
    3 & 2 & 24 & 1 028 & 5 595 & 11 211 & 14 537 & 10 580 & 2 136\\
    \hline
    4 & 2 & 24 & 1 028 & 50 419 & 305 647 & 1 022 711 & 2 835 515 & 5 291 722 \\
    \hline
    n/j & 9 & 10 & 11 & 12 & 13 & 14 & 15 & 16 \\
    \hline
    4 & 6 744 831 & 6 334 902 & 4 414 937 & 2 707 073 & 1 472 277 & 606 946 & 301 041 & 58 316\\
    \hline
  \end{tabular}
  \caption{distribúcia minimálnych DKA pre nájdené minimálne NKA pre $G_2(n)$}
  \label{tab:G1n}
\end{table}

