\chapter{Základné pojmy a definície}

Celá práca sa bude venovať konečným automatom. Najskôr si definujeme základné pojmy, uvedieme označenia a dokážeme potrebné vety, aby sme mohli ďalej na nich postaviť naše úvahy. Ako zdroj poslúžila kniha \cite{hopcroft}, resp. bakalárska práca \cite{petruchova}, ktorá sa tiež zaoberala konečnými automatmi. Ak by si čitateľ potreboval objasniť najzákladnejšie pojmy ako abeceda, gramatika, jazyk, prípadne ďalšie, tak odporúčame \cite{skripta}. 

\section{Deterministický konečný automat}

\begin{defn}{\textbf {\textit {Deterministický konečný automat (skrátene DKA)}}} A je pätica $(K,\Sigma,\delta,q_0,F)$, kde K je konečná množina stavov, $\Sigma$ je konečná vstupná abeceda, $q_0 \in K$ je počiatočný stav, $F \subseteq K$ je množina (konečných) akceptačných stavov a $\delta: K \times \Sigma \rightarrow K$ je prechodová funkcia ($\delta$-funkcia).  
\end{defn}

\begin{defn}{\textbf {\textit {Konfigurácia}}} deterministického konečného automatu je prvok $(q,w) \in K \times \Sigma*$, kde q je stav automatu a w je nespracovaná časť vstupného slova.\end{defn}

\begin{defn}{\textbf {\textit {Krok výpočtu}}} deterministického konečného automatu A je relácia $\vdash_A$ na konfiguráciách definovaná $(q,av)\vdash_A(p,v) \Longleftrightarrow p=\delta(q,a)$.\end{defn}

\begin{defn}{\textbf {\textit {Jazyk}}} akceptovaný deterministickým konečným automatom A je množina $L(A) = \{w | \exists q_F \in F; (q_0,w) \vdash^*_A(q_F,\epsilon)\}$.\end{defn}

\begin{defn}{\textbf {\textit {Minimálny DKA}}}. Deterministický konečný automat
\\ 
$A=(K_A,\Sigma,\delta_A,q_{0A},F_A)$ je minimálny, pokiaľ pre všetky DKA $B = (K_B,\Sigma,\delta_B,q_{0B},F_B)$ platí:
\\
\centerline{$L(A) = L(B) \Rightarrow |K_A| \leq |K_B|$}
\end{defn}

\label{myhillNerode}
\begin{thm}(Myhill-Nerode) Majme jazyk $L \subset \Sigma^*$. Nasledujúce tvrdenia sú ekvivalentné:
\begin{itemize}
  \item L je regulárny jazyk
  \item L je zjednotením niekoľkých tried ekvivalencie nejakej sprava invariantnej relácie ekvivalencie konečného indexu
  \item Relácia $R_L$ definovaná $uR_Lv \Longleftrightarrow (\forall x;ux \in L \Longleftrightarrow vx \in L)$ je reláciou ekvivalencie konečného indexu.
\end{itemize}
\end{thm}

Táto veta nám hovorí, že každý regulárny jazyk možno ,,rozbiť'' na niekoľko, resp. konečný počet tried ekvivalencie, pričom jeden z dôsledkov je, že minimálny DKA má práve toľko stavov, koľko tried ekvivalencie má relácia $R_L$. Ďalší dôsledok je, že minimálny DKA k jazyku L je jednoznačný až na izomorfizmus, t.j. až na pomenovanie stavov musia byť dva ekvivalentné minimálne DKA totožné. Dôkaz Myhill-Nerodovej vety nájdete napr. v \cite[Veta 2.9.1]{skripta} a jednoznačnosť minimálneho DKA v \cite[Veta 4.26]{hopcroft}.

\paragraph{}
Je dobré si uvedomiť, že budeme uvažovať len deterministické automaty s úplnou $\delta$-funkciou, t.j. nemôže sa stať, žeby v niektorom stave nebol definovaný prechod na niektorý znak. Prerobiť automat s neúplnou $delta$-funkciou na automat s úplnou je v každom prípade triviálne - stačí pre nedefinované prechody pridať ,,odpadový'' stav, čiže neprinášajú nejaké zásadné zmenšenie počtu stavov oproti tým s úplnou $\delta$-funkciou. Pri návrhu deterministických automatov sa na túto formalitu častokrát zabúda a odpadový stav sa zamlčí, to si ale my nemôžeme dovoliť, pokiaľ chceme skutočne korektne zmerať nárast počtu stavov oproti nedeterministickým konečným automatom, ktoré teraz ideme definovať.

\section{Nedeterministický konečný automat}

\begin{defn}{\textbf {\textit {Nedeterministický konečný automat}}} A je pätica $(K,\Sigma,\delta,q_0,F)$, kde K je
konečná množina stavov, $\Sigma$ je konečná vstupná abeceda, $q_0 \in K$ je počiatočný stav, $F \subseteq K$ je množina akceptačných (koncových) stavov a $\delta: K \times (\Sigma \cup \{\epsilon\}) \rightarrow 2^{K}$ je prechodová funkcia \end{defn}

\begin{defn}{\textbf {\textit {Krok výpočtu}}} deterministického konečného automatu A je relácia $\vdash_A$ na konfiguráciách definovaná $(q,av)\vdash_A(p,v) \Longleftrightarrow p \in \delta(q,a)$.\end{defn}

Konfigurácia a jazyk akceptovaný NKA sú definované rovnako ako pre DKA. Okrem toho stojí za zmienku, že minimálny NKA pre daný regulárny jazyk nie je jednoznačný, narozdiel od DKA, teda môže existovať viacero možných minimálnych ekvivaletných NKA, ktoré nie sú navzájom izomorfné.


\section{Determinizácia NKA}

\begin{defn}{\textbf {\textit {Determinizácia konečného automatu}}} $A = (K_A,\Sigma,\delta_A,q_{0},F_A)$ je DKA $D(A) = (K_{D(A)},\Sigma,\delta_{D(A)},q'_{O},F_{D_A})$, pre ktorý platí, že $L(A) = L(D(A))$.
\end{defn}

Je známy pomerne jednoduchý algoritmus na determinizáciu NKA. Po anglicky sa mu hovorí "subset construction". K NKA $A = (K_A,\Sigma,\delta_A,q_0,F_A)$ zostrojíme DKA $D = (K_D,\Sigma,\delta_D,\{q_0\},F_D)$ tak, že jeho stavy budú reprezentovať jednotlivé podmnožiny $2^{K_A}$, kde $K_A$ je množina stavov konečného automatu A. Za počiatočný stav si zvolíme množinu prislúchajúcu počiatočnému stavu a následne budeme skúmať podobne, ako pri prehľadávaní do širky, na ktoré stavy v tom pôvodnom NKA A sa môžeme dostať cez daný znak. Konečné stavy potom budú tie, ktoré reprezentujú množiny obsahujúce niektorý z konečných stavov v pôvodnom NKA A. Formálne: 
\\
\centerline {$K_D = 2^{K_A}$}
\\
\centerline {$\forall a \in \Sigma, S \subseteq K_A: \delta_D(S,a) = {\bigcup}_{p \in S} \delta_A(p,a)$}
\\
\centerline {$F_D = \{S: (S \in K_D) \wedge (S \cap F_A \neq \emptyset)\}$}
\\
Samozrejme, pri samotnej implementácii môžeme zanedbať nedosiahnuteľné stavy, ktorých pravedpodobne pre daný konečný automat bude dosť veľa. V prípade, žeby čitateľ chcel vedieť viac podrobností, odporúčame \cite[Kapitola 2.3.5]{hopcroft}.


\section{Stavová zložitosť regulárnych jazykov}

Našim hlavným cieľom je, ako názov práce napovedá, analyzovať ,,deterministickú a nedeterministickú stavovú zložitosť regulárnych jazykov''. Definujme si teda formálnejšie tieto pojmy:

\begin{defn}{\textbf {\textit {Deterministická stavová zložitosť}}} regulárneho jazyka L je počet stavov minimálneho DKA vzhľadom na počet stavov, ktorý tento jazyk akceptuje.
\end{defn}

\begin{defn}{\textbf {\textit {Nedeterministická stavová zložitosť}}}  regulárneho jazyka L je počet stavov minimálneho NKA vzhľadom na počet stavov, ktorý tento jazyk akceptuje.
\end{defn}

Aby sme sa nemuseli vždy vyjadrovať takto zdĺhavo, tak nedeterministickú a deterministickú stavovú zložitosť budeme skrátene označovať ,,nedeterministická'', resp ,,deterministická zložitosť jazyka''.